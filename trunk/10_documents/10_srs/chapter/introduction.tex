\section{Introduction}

	\subsection{Purpose}
	This document describes the requirements of RowBuddy, a Web based system to log routes of rowing boats at the Crefelder Ruder-Club 1883 e.V. (Crefelder RC). It will be developed by students of the JEE course in the seventh semester of the study course Software Engineering at the Fontys University of Applied Sciences, Venlo / The Netherlands.\\
	
	This document is addressed to the system developers and as well to the lecturer of the JEE course at the Fontys University of Applied Sciences.
	
	
	\subsection{Scope}
	\label{scope}
	The purpose of RowBuddy is to provide a web based software system to offer the possibility to log rowed trips, concerning a special boat, the rowers and also the route.\\
	
	The Crefelder RC owns two boat houses and each house offers an own logbook for rowed courses. For statistical evaluation it is very difficult to combine the logged entries, which normally is done at the end of each year. Because the Crefelder RC consists of over 300 members and has many different boats, at least more than 50, it is very important to log every trip that is rowed. Also for insurance it is necessary to offer this possibility to every rower.\\
	
	The actual situation with two different log books is found inefficient for the following main reasons:
	\begin{itemize}
		\item Inefficient statistical work at the end of each year
		\item No standard routes are available, which are rowed multiple times every day
		\item Double logged entries may occur
	\end{itemize}
	
	RowBuddy will offer the possibility to delete these inefficient points. Firstly it will offer the possibility to \textbf{save the boats, the members of the rowing club, standard routes and of course it will log rowed events}. Secondly the big improvement of RowBuddy will be the \textbf{statistical views}, the software system will offer.\\
	Another big improvement will be the possibility to offer profiles of each member of the rowing club where logged routes are showed to visitors of his profile.\\
	
	Six main parts can be identified:
	\begin{itemize}
		\item An online software system to store all members of the club
		\item A module to manage boats, reservations and damages
		\item A module to manage routes that can be rowed
		\item A tool to log the the rowed trips of members
		\item An automatic statistic generation tool
		\item A profile for each club member where he can present his rowed courses
	\end{itemize}
	
	\subsection {Definitions, acronyms and abbreviations}
	Please refer to Appendix \ref{appendix:definitions} for a list of definitions, acronyms and abbreviations.
	
	
	\subsection{References}
	Please refer to Appendix \ref{appendix:references} for all document references in this software requirements document.
	
	\subsection{Overview}
	Section 2 of this document provides a use case model survey. It does not state specific requirements. Instead it describes the known actors and their interactions with RowBuddy from a high level view. It also lists factors and assumptions, that affect the requirements stated in this document.